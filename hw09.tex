\chapter{Непропорциональное страхование. Экцедент убытка по риску/катастрофе. Финансовые и экономические условия.}
\problem{}
Рассматривается договор эксцедента убытка по риску $XL$: $5\, xs\, 2$. Предполагается, что возможны 4 возобновления. Добавочные премии за возобновление полосы: $25\%$, $50\%$, $100\%$, $200\%$.  Произошло $8$ убытков, их размеры: $5$, $10$, $7$, $4$, $6$, $8$, $3$, $9$. Первоначальная премия равна 4. 
Подсчитать размер добавочных премий. Все размеры в млн. 

\solution{}
Пусть $X_i$ -- указанные убытки, $Y:= \min\left\{5, (X_i-2)_+\right\}$ -- перестраховое покрытие, $L = 5\cdot\left(4+1\right) = 25 $ -- максимальное значение гарантий перестраховщика, $Y = \sum_{i=1}^8 Y_i$ -- суммарные выплаты по обязательствам перестраховщика.
\begin{table}[h]\centering
    \begin{tabular}{|c|c|c|c|c|c|}\hline
        $i$ & $X_i$ & $Y_i$ & $Y$ & $XL$ & добавочная премия \\\hline\hline
        1   & 3     & 3     &  3   &  3   & 0.6 \\\hline
        2   & 10    & 5     &  8   &  5   & 1.6 \\\hline
        3   & 7     & 5     &  13  &  5   & 3.2 \\\hline
        4   & 4     & 2     &  15  &  2   & 1.6 \\\hline
        5   & 6     & 4     &  19  &  4   & 6.4 \\\hline
        6   & 8     & 5     &  24  &  5   & 1.6 \\\hline
        7   & 3     & 1     &  25  &  1   & --- \\\hline
        8   & 9     & 5     &  30  & ---  & --- \\\hline
    \end{tabular}
\end{table}

Добавочные премии закончились на $7$-м убытке, т.к. мы достигли максимальных гарантий перестраховщика.






\problem{}

Подсчитать, чему равна премия по договору $3 \,xs\, 2$ (млн.), если размеры последовательных убытков равнялись $3$, $3.4$, $3.2$, $4.8$, $4.4$, $7$. 
Предполагается, что применяется скользящая ставка премии от $2\%$ до $5\%$ (при коэффициенте надбавки $100/80$ убытков на гарантии 
перестраховщика, уже оплаченных или еще не урегулированных). Премия прямого страховщика равна $200\cdot 10^6$. 

\solution{}
Пусть $X_i$ -- указанные убытки, $Y:= \min\left\{3, (X_i-2)_+\right\}$ -- перестраховое покрытие, $Y = \sum_{i=1}^6 Y_i = 11.8$.
\begin{equation}
    r = \min\{\underbrace{r_{\max}}_{5\%}, \underbrace{\max\{\underbrace{r_{\min}}_{2\%}, \underbrace{\tfrac{dY}{A}}_{=\frac{100\times 11.8}{8\times 200} = 73.75\%}\}}_{73.75\%}\} = 5\%.
\end{equation}
Итого, $P = rA = .05\times 200 = 10$.