\chapter{Оптимальное перестрахование. Порядки рационального перестраховщика, эксцедента богатства и рассеивания.}
\problem{}

\solution{}
\begin{itemize}
    \item Пусть $X \sim U[0,2b]$. Тогда
    \begin{multline}
        \pi_\rho(X) = \int_0^{2b} (1 - t/(2b) ) ^{\frac{1}{\rho}} dt =\\= [ 1 - t/(2b) = y, \; -2bdy = dt] = 2b\int_0^1 y^{\frac1\rho} dy = 2b \frac{1}{\frac{1}{\rho} + 1} = \frac{2b\rho}{1 + \rho}.
    \end{multline}
    Подставляя значения $\rho$ получаем $\pi_{1,2}(X) = \frac{2.4 b}{2.2} \approx 1.09b$, $\pi_{1,5}(X) = \frac{3 b}{2.5} = 1.2b$, $\pi_{1,8}(X) = \frac{3.6 b}{2.8} \approx 1.286b$
    \item Пусть $Y \sim \exp(1/b)$. Тогда 
    \begin{equation}
        \pi_\rho(Y) = \int_0^{+\infty} \exp(-t/(b\rho)) dt = b\rho .
    \end{equation}
     Подставляя значения $\rho$ получаем  $\pi_{1,2}(Y) = 1.2b$, $\pi_{1,5}(Y) = 1.5b$,  $\pi_{1,8}(Y) = 1.8b$.

     \item Пусть $Z \sim \bar F_Z(t) = b^2 /(b+t)^2$. Тогда 
     \begin{multline}
         \pi_\rho(Y) = \int_0^{+\infty} b^{2\rho}/(b+ t)^{2\rho} dt =\\= [ y = b + t, dy = dt] = b^{2\rho}\int\limits_{b}^{+\infty} y^{-2\rho}dy = \frac{b}{2\rho - 1}, \; \rho>1/2.
     \end{multline}
     Подставляя  значения $\rho$ получаем $\pi_{1,2}(Z) = b/1.4$, $\pi_{1,5}(Z) = b/2$, $\pi_{1,8}(Z) = b/2.6$.
\end{itemize}