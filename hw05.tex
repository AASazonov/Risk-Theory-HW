\chapter{Выпуклый порядок. Свойства инвариантности стоп-лосса. Сравнение биномиальной, пуассоновской и отрицательной биномиальной моделей.}
    \problem{}
        Пусть $<_b$ - полный порядок всех рисков для лица $b$ из некоторого множества $B$ лиц, принимающих решения. Определим бинарное отношение $<_a$ на множестве рисков следующим образом: $X<_a Y$ тогда и только тогда, когда $X<_bY$ для всех $b \in B$. Доказать, что $<_a$ - частичный порядок, (отражающий предпочтения всех лиц из мн-ва $B$).
        \solution{}
            Необходимо проверить свойства рефлексивности, транзитивности и антисимметричности. 
            \begin{itemize}
                \item[\textbf{Рефлексивность}] $X<X$. Возьмем произвольное $b \in B$. $X<_b X$ - поскольку $<_b$ - рефлексивно. В силу произвольности $b \in B$ получаем, что $X<_a X$.
                \item[\textbf{Транзитивность}] $X<Y,\; Y<Z \Rightarrow X<Z$.
                Пусть $X<_aY,\; Y<_a Z$. Значит $X<_b Y \; \forall b \in B$ и $Y<_b Z \; \forall b \in B$. Возьмем произвольное $b \in B$. Тогда, поскольку $<_b$ - полный порядок на множестве рисков для лица $b$, то $X<_b Z$ по транзитивности для порядка $<_b$. В силу произвольности $b \in B$ получаем, что $X<_a Z$.
                \item[\textbf{Антисимм-ть}] $X<Y, \; Y<X \; \Rightarrow X=Y$. Пусть $X<_aY$ и $Y<_aX$. Т.е. $X<_bY$ и $Y<_bX$ $\forall b \in B$. Возьмем произвольное $b \in B$. Из $X<_b Y$ и $Y<_b X$ следует, что $X=Y$ из антисимметричности $<_b$. В силу произвольности $b \in B$ получаем, что из $X<_aY$ и $Y<_aX$ следует, что $X=Y$.
                
            \end{itemize}
            Итак, мы доказали, что отношение $<_a$ обладает свойствами транзитивности, рефлексивности и антисимметричности. Значит отношение $<_a$ является частичным порядком на множестве всех рисков, который учитывет предпочтения всех лиц $b$ из множества $B$.
    
    \problem{}
        Предположим, что $X_1 <_{st} X_2$. Можно ли (на том же самом вероятностном пространстве) найти такую случайную величину $X^\prime_2$, чтобы $X_1 <_1 X^\prime_2$ и $X_2 \stackrel{d}{=} X^\prime_2$ ?
        \solution{}
            Определим вероятностное пространство следующим образом: $\Omega=\{0,1\},\; \mathcal{F} = 2^\Omega, \mathbb P(\{0\}) = \frac34, \mathbb P(\{1\}) = \frac14, \mathbb P(\Omega) = 1, \mathbb P(\varnothing) = 0$. Определим случайные величины как $X_2 = \textbf{1}_{\{0\}}(\omega), \; X_1 = \textbf{1}_{\{1\}}(\omega)$, где $\textbf{1}_{\{i\}}(\omega)$ - индикатор точки $i$. Легко видеть, что $X_1 <_{st} X_2$. Однако, не существует $X^\prime_2 \stackrel{d}{=} X_2$, т.ч. $X_1 <_1 X^\prime_2 $. Действительно, пусть существует. Тогда $X^\prime_2(1) \geq 1 = X_1(1)$. Поскольку $\mathbb P(X^\prime_2 = 0) = \mathbb P (X_2 = 0) =\frac14$ то $X^\prime_2(1) = 0$, так как $\mathbb P(A) = \frac{1}{4} \; \Leftrightarrow \; A=\{1\} $. Противоречие.
            
    \problem{}
        Доказать свойство $3^\circ$ непосредственно, пользуясь определением свертки. Будут ли выполнены свойства $1^\circ,2^\circ,4^\circ$ для стохастического порядка?
        \solution{}
            Свойство $3^\circ$: $F_1 \prec F_2, \; G: \; F_k*G \in B_{\prec},\; k=1,2 \; \Rightarrow \; F_1*G \prec F_2*G $.
                
            Пусть $F_1 <_{st} F_2$. Тогда $F_1(t) \geq F_2(t), \; \forall t  $. Тогда получаем, что 
            \begin{equation*}
                \int_{-\infty}^{\infty} \textbf{1}(x+y\leq t) dF_1(x) \geq \int_{-\infty}^{\infty} \textbf{1}(x+y\leq t) dF_2(x) , \; \forall y \in \mathbb R,
            \end{equation*}
            поскольку вариация функции $F_1(t)$ с учетом знака на луче $(-\infty, z]$ больше или равна вариации функции $F_2(t)$ на том же луче $\forall z \in \mathbb R$.
            Поскольку $G(t)$ - монотонно неубывает на всей прямой, то интеграл римана-стилтьесса сохраняет неравенства, т.е. 
            \begin{equation*}
                \int_{-\infty}^{\infty}\int_{-\infty}^{\infty} \textbf{1}(x+y\leq t) dF_1(x)dG(y) \geq \int_{-\infty}^{\infty}\int_{-\infty}^{\infty} \textbf{1}(x+y\leq t) dF_2(x)dG(y) , \; \forall y \in \mathbb R,
            \end{equation*}
            что равносильно 
            \begin{equation*}
                G*F_1 <_{st} G*F_2.
            \end{equation*}
    
    \problem{}
        Сохраняется ли стохастический порядок при взятии составных распределений?
        \solution{}
        Пусть $\{X_k\}_{k=1}^\infty$ - н.о.р., $N$ - неотрицательная, целочисленная случайная величина не зависящая от $\{X_k\}_{k=1}^\infty$, $p_k = \mathbb P(N=k)$. $S_n = \sum_{k=1}^n ,\; S_0 = 0$. Тогда 
        \begin{equation*}
            P(S_N\leq x) = \sum_{k=0}^\infty p_k\mathbb P(S_k \leq x) = \sum_{k=0}^\infty p_kF^{*k}(x).
        \end{equation*}
        Из задачи $2$ и замечания к задаче 2 следует, что если $F_1 <_{st}F_2$, то $F_1^{*k} <_{st} F_2^{*k}$. Действительно, для $k=1$ - это верно. Пусть верно для $k=n-1$, докажем для $k=n$. 
        \begin{equation*}
           F_1^{*(n-1)}<_{st} F_2^{*(n-1)} \; \Longrightarrow \; F_1^{*(n-1)}*F_2 <_{st} F_2^{*n}.
        \end{equation*}
        \begin{equation*}
            F_1<_{st} F_2 \; \Longrightarrow \; F_1^{*n} <_{st} F_1^{*(n-1)}*F_2
        \end{equation*}
        Пользуясь транзитивностью стохастического порядка получаем требуемое утверждение.
        
        Из определения стохастического порядка видно, что если $\sum_{k=0}^n p^n_k = 1,\; p^n_k\geqslant0$ и $F_{1,k}k <_{st} F_{2,k} \; \forall k$, то $\sum_{k=0}^np^n_kF_{1,k} <_{st} \sum_{k=0}^np^n_kF_{2,k} $, или, что тоже самое (по определению стохастического порядка) 
        \begin{equation}
            \sum_{k=0}^np^n_kF_{1,k}(t) \geq \sum_{k=0}^np^n_kF_{2,k}(t),\; \forall t\in \mathbb R.
        \end{equation}
        
        Переходя к пределу по $n$ в последнем неравенстве, получаем 
        \begin{equation*}
            \sum_{k=0}^\infty p_kF_{1,k}(t) \geq \sum_{k=0}^\infty p_kF_{2,k}(t),\; \forall t\in \mathbb R,
        \end{equation*}
        или, равносильно
        \begin{equation*}
            \sum_{k=0}^\infty p_kF_{1,k} <_{st} \sum_{k=0}^\infty p_kF_{2,k}.
        \end{equation*}
        Применяя все написаное выше непосредственно к нашей задаче, получаем: если $F_1 <_{st} F_2$, то
        \begin{equation*}
            \sum_{k=0}^\infty p_kF_1^{*k} <_{st} \sum_{k=0}^\infty p_kF_2^{*k}.
        \end{equation*}
        Итак, получаем следующее утверждение: пусть $\{X_k^i\}_{k=1}^\infty$ - н.о.р. с функцией распределения $F_i(x)$, $i=1,2$, $N$ - неотрицательная целочисленная с.в. . Пусть также $F_1 <_{st}F_2$. Определим $S^i_n=\sum_{k=1}^n ,\; S_0^i = 0$, $G_i(x) = \mathbb P(S_N \leq x) ,\; i=1,2 $. Тогда $G_1 <_{st} G_2$, т.е. взятие составного распределения сохраняет стохастический порядок. 
        
    \problem{}
        Если $\mathbb EX = m$, то $m<_{sl} X$.
        \solution{}
            Покажем, что $\mathbb E \max(d,m) \leq \mathbb E \max(d, X)$, что эквивалентно $m<_{sl}X$. Пусть $d>m$. Тогда $\max(d,m) = m$. Тогда $d\leq \max(d, X)$ и, следовательно $\mathbb Ed \leq \mathbb E\max(d,X)$. Пусть $m\geq d$. Тогда, поскольку $X\leq \max(d,X)$ и $\max(m,d) = m$,то $ m=\mathbb EX\leq \mathbb E \max(X,d)$. Итак, получили что $\forall d \in \mathbb R \;\;\; \mathbb E \max(d,m) \leq \mathbb E \max(d,X)$, т.е. $m<_{sl}X$.
            
