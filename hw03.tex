\chapter{Суммарный размер ущерба. Динамические модели. Модель Крамера - Лундберга и вероятность разорения}
    \problem{}
        Показать, что логарифмически нормальное распределение масштабно инвариантно, но не обладает масштабным параметром.
    \solution{}
        \begin{definition}[Масштабно инвариантное семейство]
            Семейство распределений называется масштабно инвариантным, если вместе с распределением случайной величины $X$ распределение $Y=cX \quad \forall c \in \mathbb{R^+}$ также принадлежит этому семейству.  
        \end{definition}
        $X\sim LN(a, \sigma^2)$, т.е. $\exists Y \sim N(a, \sigma^2)\colon \ X = \exp \left\{ Y \right\}$.
        Посмотрим, какое распределение у $cX$:
        \begin{equation*}
            cX = c\exp\left\{Y\right\} = \exp\left\{Y + \log c\right\},
        \end{equation*}
        т.е. $cX \sim LN(a+\log c, \sigma^2)$.

    \problem{}
        Доказать, что пуассоновское распределение $Pois(\lambda)$ получается из отрицательно биномиального распределения $NB(\alpha, \beta)$, если положить $\lambda = \alpha(1 - \beta)$ и $ \beta \to 1$.
    \solution{}
        \begin{definition}
            [Отрицательное биномиальное распределение $NB(m, p)$]
                \begin{equation*}
                        \P(X = k) =\tbinom{m+k-1}{k} p^m(1-p)^k \iff \MGF_X(t) = \left(\frac{1-p}{1-pe^t}\right)^m.
                \end{equation*}
        \end{definition}
        Пусть $X\sim NB(\alpha, \beta), \alpha\beta=:\lambda$. Найдем предельную $MGF$ при $\beta\to 0$:
        \begin{multline*}
            \lim_{\beta\to 0}\MGF_X(t) = \lim_{\beta\to 0} \left(\frac{1-\beta}{1-\beta e^t}\right)^\alpha = \\ =
            \lim_{s\to 0}  \left(\frac{1-s}{1-se^t}\right)^{\lambda / s} = 
            \lim_{s\to 0}  \left(\frac{1-se^t}{1-s}\right)^{-\lambda / s} \overset{\text{из 2 зам.пр.}}{=\!=\!=\!=\!=\!=} e^{\lambda \left( e^{t}-1 \right) } = \MGF_{Pois(\lambda)}(t).
        \end{multline*} 
        По теореме о характеризации получаем, что предельное распределение пуассоновское ($\MGF \leftrightarrow \law$).

    \problem{}
        Все ли распределения класса $(a, b, 0)$ являются безгранично делимыми?
    \solution{}
        \begin{definition}
            Считающее распределение принадлежит классу $(a, b, 0)$, если 
            \begin{equation*}
                p_k = p_{k-1}\left( a+\frac{b}{k} \right).
            \end{equation*}
            При этом $p_0 = 1-\sum_{k = 1}^\infty p_k$
        \end{definition}
        Знаем, что только $Pois, B, NB, Geom$\footnote{$Geom$ частный случай $NB$} принадлежат этому классу.
        \begin{definition}
            Случайная  величина $Y$ (ее распределение) называется бесконечно делимой (-ым), если для любого $n \in \mathbb{N}$ она может быть представлена в виде $Y = \sum_{i=1}^n X^{(n)}_i$, где  $(X_i^{(n)})_{i=1\dots n}$ -- независимые одинаково распределённые случайные величины.
        \end{definition}
        \begin{itemize}
            \item $Pois(\lambda)$: берем $n$ н.о.р. $Pois(\lambda/n)$ случайных величин;
            \item $B(m, p)$ \begin{equation*}
                \MGF_{B(m, p)} = \left(1-p + pe^t\right)^m = \left(\left(\sqrt[n]{1-p + pe^t}\right)^m\right)^n;
            \end{equation*}
            \item $NB(m, p)$ \begin{equation*}
                \MGF_{NB(m, p)} = \left(\frac{1-p}{1-pe^t}\right)^m = \left(\left(\sqrt[n]{\frac{1-p}{1-pe^t}}\right)^m\right)^n.
            \end{equation*}
        \end{itemize}
    \problem{}
        Выписать явный вид $(p^T_k)_{k\geq 1}$ для урезанных в нуле распределения из класса $(a, b, 0)$.
    \solution{}
        $p^T_k = \frac{p_k}{(1 -p_0)}$. Далее $k\geq 1$.
        \begin{itemize}
            \item $TB(n, p)$
            \begin{equation*}
                p_k^T = \binom{n}{k} \frac{p^k(1-p)^{n-k}}{1 - (1-p)^n};
            \end{equation*}
            \item $TPois(\lambda)$
            \begin{equation*}
                p^T_k = \frac{1}{k!} \frac{\lambda ^ k }{e^\lambda - 1};
            \end{equation*}
            \item $TNB(\alpha, \beta)$
            \begin{equation*}
                p_k^T = \tbinom{k + \alpha - 1}{k}\frac{\beta^{k}}{(1+\beta)^{\alpha+k} - (1+\beta)^k};
            \end{equation*}
            \item $TGeom(\beta)$
            \begin{equation*}
                p_k^T = \frac{\beta^{k - 1}}{(1 + \beta)^k}.
            \end{equation*}
        \end{itemize}